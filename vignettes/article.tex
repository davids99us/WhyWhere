\documentclass[10pt]{article}
\usepackage{times}
\usepackage{amsfonts}
\usepackage{hyperref}
%\pagenumbering{gobble}
\usepackage{float}
\usepackage{setspace}
\floatplacement{figure}{H}
\usepackage{graphics}
%\VignetteIndexEntry{Using WhyWhere}
\newcommand{\pkg}[1]{{\fontseries{b}\selectfont #1}} 
\usepackage{listings}
\lstdefinestyle{myCustomMatlabStyle}{
  language=R,
  stepnumber=1,
  basicstyle=\footnotesize,  
  tabsize=4,
  showspaces=false,
  showstringspaces=false
}
\DeclareTextFontCommand{\emph}{\bfseries}
\usepackage{geometry}
\geometry{
left=20mm,
right=20mm,
top=25mm,
bottom=15mm}
\usepackage{Sweave}
\begin{document}
\Sconcordance{concordance:article.tex:article.Rnw:%
1 11 1 1 0 34 1 1 2 1 0 1 2 1 0 5 1 3 0 1 2 25 1 1 2 4 0 1 2 104 1}


{\setstretch{1.5}
\begin{center}
\textsf{\Large \textbf{Segmented or generalised regression models in predicting and explaining species distributions}}

\textsf{\large David R.B. Stockwell$^1$}


\textsf{ $^1$Centre for Intelligent and Networked Systems, Central Queensland University, Australia. \\
Email: d.stockwell@cqu.edu.au}
\end{center}

\textsf{ \textbf{Keywords:} machine learning, big data, species distributions, fuzzy sets}}
\vspace{5mm}

\textbf{Abstract.} Previous studies have suggested that machine learning approaches utilizing large datasets of environmental variables might be an efficacious approach to species distribution modelling.  In this study we demonstrate the application of a new version of \texttt{WhyWhere} against a generalized linear model on $Bradypus~variegatus$ (the brown-throated three-toed sloth) on two data sets: a 9 variable WorldClim climatic data sets and a 940 global data sets of mixed type on the Global Ecosystems Database.  The primary measure of performance was the area under the curve of the receiver operating characteristic.  We demonstrate that \texttt{WhyWhere} has superior accuracy at predicting Bradypus on both data sets and identifies a number of accurate, but completely different variables with the potential to  provide unique insights into the ecology of the species not provided by the climatic variables.  The implications of the two approaches for SDM are discussed. \texttt{WhyWhere} is available as an R package from the development site at http://github.com/davids99us/whywhere.

\section{Introduction} 

Novel ecological niche models of all kinds have proven efficacy for modelling species distributions when adequate species occurrence data, environmental correlates and modelling methods are used \cite{citeulike:585800}. Methods where the form the structure of the model resembles the unimodal response of the species to the environment such as BIOCLIM \cite{Nix:1986yg} and generalized linear models (GLMs) \cite{Austin:1996np} are typically combined with climatic variables to represent the constraints of the climatic range of the species.  In this case the use of bioclimatic variables such as the WorldClim database \cite{Hijmans2005} is associated with high predictive accuracies.   

At finer scales where there is little climatic variation, such as the study of rare and endemics, the species may be responding to habitat features with discrete structure such as nesting sites, physical barriers and resources.  Thus models need to be constructed from a wide range of variables: soils, vegetation type and vertical development, and ecoregions. For example, $Bradypus~variegatus$ (the brown-throated three-toed sloth) is a largely arboreal mammal of the Amazon and Central America regions.  While it crawls along the forest floor poorly, it does swim well \cite{Hayssen2010}.  While climate variables may provide correlations of its entire range, its proximal response may be  linked to the type of forest vegetation, weather conditions associated with that forest such as cloudiness, or seasonal inundation known a flooded forest.  

Thus it can be seen that many variables may have predictive value.   Little is known about a great many species, raising concern that suboptimal selection of model structure and environmental variables produces inferior model results.  Some points may also be errors or zoo locations which is typical of opportunistic data sets. A modelling system should be robust to errors, as the primary purpose is to provide systems that do not require advanced expertise.  These concerns have driven the agenda of general purpose modelling systems which is a long term goal of using an artificial intelligence environment \cite{Davey:1991yg}.  Moreover, the potential environmental data sets are in a variety of resolutions and projections, which can be very time consuming and error prone to rectify. It would be convenient for systems to handle this.

We hypothesized an approach in \cite{Stockwell:2006xt, Stockwell:2006rf} that a prototype system called WhyWhere (WW1) would address the above performance goals.  WW1 used segmented models to generalization over the response of the species for both continuous and categorical models.  The species modelling program MaxEnt addresses the issue of multiple response types by providing a number of potential response types: e.g. linear, quadratic categorical, logarithmic \cite{Phillips2007}.  By comparison, the segmented model in \texttt{WhyWhere} provides a single unified approach by cutting up the range of the variable into discrete categorical factors.  WW1 \cite{Stockwell:2006xt} used a median cut algorithm which assigns equal numbers of points to each color \cite{Heckbert82}.  This has been shown to retain good visual appearance, but also has the statistical justification of minimizing the variance across the range, by minimizing the variance in each category.  

Here we present the advances in algorithm structure, accuracy and explanatory capacity of a new implementation of the WhyWhere algorithm into new R package (WW2) and advances in the algorithmic flow. Models of $Bradypus~variegatus$ (the brown-throated three-toed sloth) in Amazonia use selected environmental variables in the WorldClim climatic data sets provided in the R package \texttt{dismo} and 940 global data sets of mixed type on the Global Ecosystems Database.  We show an improvement in accuracy from data-mining large numbers of environmental data sets compared with simple sets of climatic variables. It is demonstrated that the method would be robust to any non-linear continuous (e.g. temperature, rainfall) or categorical (e.g. biome or soil type).  

\section{Methods} 

\subsection{Data}

The \texttt{dismo} R package includes locations for  $Bradypus~variegatus$ and a selection of WorldClim environmental variables in its distribution package.  The Bradypus species data consist of 116 longitude and latitude records of occurrences in covering Central and South America.  Table~\ref{tab1} lists the 9 bioclimatic variables from the WorldClim and one variable from the ecoregions databases \cite{Hijmans2005, Olson2001}. These are available at a common resolution of 0.5 degrees and projection of WGA84.  The algorithm crops the environmental data to one degrees of the maximum extent the recorded locations. 



% latex table generated in R 3.2.0 by xtable 1.7-4 package
% Tue Jul  7 21:42:11 2015
\begin{table}[ht]
\centering
\begin{tabular}{rlllrl}
  \hline
 & File & Variable & Source & Resolution..deg. & Type \\ 
  \hline
1 & bio1.grd & Annual Mean Temperature & BIOCLIM & 0.50 &  real \\ 
  2 & bio12.grd &  Annual Precipitation & BIOCLIM & 0.50 &  real \\ 
  3 & bio16.grd &  Precipitation of Wettest Quarter & BIOCLIM & 0.50 &  real \\ 
  4 & bio17.grd &  Precipitation of Driest Quarter & BIOCLIM & 0.50 &  real \\ 
  5 & bio5.grd &   Max Temperature of Warmest Month & BIOCLIM & 0.50 &  real \\ 
  6 & bio6.grd &  Min Temperature of Coldest Month & BIOCLIM & 0.50 &  real \\ 
  7 & bio7.grd &   Temperature Annual Range (BIO5-BIO6) & BIOCLIM & 0.50 &  real \\ 
  8 & bio8.grd &   Mean Temperature of Wettest Quarter & BIOCLIM & 0.50 &  real \\ 
  9 & biome.grd & Terrestrial Ecoregions of the World & WWF & 0.50 &  category \\ 
   \hline
\end{tabular}
\caption{The variables in the 	exttt{dismo} package from WorldClim with resolution and type.} 
\label{tab1}
\end{table}
% latex table generated in R 3.2.0 by xtable 1.7-4 package
% Tue Jul  7 21:42:11 2015
\begin{table}[ht]
\centering
\begin{tabular}{rlllrl}
  \hline
 & File & Variable & Source & Res.deg. & Type \\ 
  \hline
1 & fnocwat.txt &  Navy Terrain Data--Percent Water Cover &  GED & 0.17 &  real \\ 
  2 & lccld07.txt &  Leemans and Cramer July Cloudiness (\% Sunshine) &  GED & 0.50 &  real \\ 
  3 & i00an1.1.pgm &  World Ocean Atlas 2001 - silicate at depth 0 metre  &  GED & 1.00 &  real \\ 
  4 & etopo.pgm &  Elevation from the National Geographic Data Center &  GED & 0.03 &  metres \\ 
   \hline
\end{tabular}
\caption{Selection of the 940 variables from the Global Ecosystems Database used in this study with resolution and type.} 
\label{tab2}
\end{table}
The second environmental dataset was the Global Ecosystems Database (GED) and consists of 940 global data sets of mixed type on the Global Ecosystems Database (selected on Table~\ref{tab2}). These are all of global extent with a range of resolutions from 1 deg to 0.033... degrees and in the same global geographic projection.   They were scaled to fit 0-255 values for the WW1 project in order to be more compact.  This is no longer necessary and future studies will use the native datasets.    


\subsection{Algorithms}
 

% latex table generated in R 3.2.0 by xtable 1.7-4 package
% Tue Jul  7 21:42:11 2015
\begin{table}[ht]
\centering
\begin{tabular}{ll}
  \hline
GLM & WW2 \\ 
  \hline
1. Input long and lat occurrences &  1. Input long and lat occurrences \\ 
  2. Presample background points &  $<$optional$>$ \\ 
  3. Input all env. vars &  2. Start loop through list of env. vars.  \\ 
  5. Create flat file & 2.1 Input env. variable \\ 
  4. Fit additive model &  2.2 Fit segmented model \\ 
  5. Output model &  2.3 $<$optional$>$ Evaluate conjunction with best model so far \\ 
  6. Evaluate model &  3. Output ordered list of models evaluated \\ 
   \hline
\end{tabular}
\caption{ The flow of analysis for GLMs and WW2 shows the stages of data inputs, fitting, and evaluation.} 
\label{tab3}
\end{table}
The flow of analysis for GLM and WW2 is shown on Table~\ref{tab3}. The stages include data inputs, fitting, and evaluation.  All occurrences are given as $<x,y>$ pairs of longitude and latitude. The data outputs of both methods include a probability estimate at each environmental data point. Other outputs include the respective models, other possible models, both singly and in combination, and maps of the probability of species in the region of interest, and so on.
 
 The GLM  requires a flat file in so-called 'wide' format.  The wide format is where the variables are in columns and the data points are in rows. These data include the presence or absence and a listing of all environmental value(s): 

\begin{equation}
<x,y,pa,v_1,...,v_n>
\end{equation}

where $x$ is longitude, $y$ is latitude, $pa$ is dependent variable the presence of absence of the species, and $v_1,...v_n$ are the independent variables which are the values at each of the locations $<x,y>$. In the case of presence-only data, the environmental values are generally drawn from a sample $B$ of the possible environmental data sets $G$, generally referred to as backgrounding.  This was developed as a means to provide pseudo-absence data when none exists -- such as in the case of museum data -- and so allow the use of these techniques.


The default inputs to WW2 are the longitude and latitude of known locations and a set of environmental data files $G_i$ that may or may not be coregistered.  They must be able to be read into the \texttt{raster} package.  The model in WW2 is based on the comparison of the frequency of environmental values in the environment $G$ and the frequency of environmental values in the given locations $S$.  The model is in effect a lookup table on segments in the environmental variable.  As WW2 makes use of the frequency of the environmental values $G$ and does not need a 'wide' file, even in multidimensional analysis, backgrounding is optional.  The WW2 algorithm can predict on both presence-only and presence-absence data.  Parameters to WW2 include \emph{multi} identify the number of conjunctions of variables to be searched.  

The data outputs of both methods is a probability estimate at each environmental data point. Other outputs include the respective models, other possible models, both singly and in combination, and maps of the probability of species in the region of interest, and so on.

The main loop of the algorithm develops a segmentation of the variables and a lookup table based on that segmentation.  The output is a table listing each model tested ranked in decreasing order by the Chi-squared value.  A standard Chi-squared test compares the counts of values of pixels in each segment of the overall environment$(G)$, against the count of values in those pixels where the species occurs $(S)$.  A significant difference in the distribution is indicative of a strongly biased sample indicative of the response of the species to its environment. 

In the current implementation of a multi-dimensional model, only the best variable is combined with each new variable using the fuzzy minimum of the predicted probabilities at each point. Alternative approaches to searching the space of conjunctions may be implemented in future.  It is possible to monitor the progress with the plot option in a streaming work flow.

\subsection{Response function}

In a second or third order GLM the response of species is a 'humped' function on the range of the environmental variable.  This function represents the falling frequency of the species around an ideal habitat, or restriction to a range of a variable (e.g. the range of temperature tolerances).  While often second order or quadratic it must be at least order three to incorporate skewness. The range of the GLM is the odds of occurrence of the species at each environmental value where the species occurs.

In WW2 strength of a species' response is given by the frequency of environmental value in each segment of the environmental value where the segment is determined by a quantile. WW2 segments a single continuous or categorical variables into equal quantile open-closed intervals. The number of segments $2n$ is determined using the Freedman-Diaconis rule \cite{Freedman1981} for optimal binning of histograms.  WW1 \cite{Stockwell:2006xt} used a median cut algorithm which assigns equal numbers of points to each color \cite{Heckbert82}.  This has been shown to retain good visual appearance, but also has the statistical justification of minimizing the variance across the range, by minimizing the variance in each category.  

\begin{equation}
G_i = (v_1,v_2]_1, ... , (v_{n-1},v_n]_j , ... , (v_{2n-1},v_2n]_n
\end{equation}

For a given environmental variable $G_j$ and a segmentation, the tabulation of the counts in each of the environmental values allows calculation of the frequency of values in each segment $j$ where the species occurs and over the prior frequency of environmental values. These frequencies are $s_j$ and $g_j$ respectively from which we calculate the odds ration or $OR$.

\begin{equation}
OR_j = s_j (1-g_j) / g_j (1 - s_j)
\end{equation}

The odds ratio is the increase in frequency of a species occurring in an environment over a specific range $S$, relative to the naturally occurring frequency of that environmental range in the region $G$.  Given the odds ratio, the response is given as an expected probability $P: \mathbb{R} \rightarrow [0,1]$ where 

\begin{equation}
P(OR_j)_j =OR_j/(1+OR_j) 
\end{equation}

While we would like to estimate the probability of species being present using Bayes' Theorem, it is dependent on season, search effort and other uncontrolled variables.  The proxy above is sufficient and useful in most applications \cite{Stockwell:1993yg,Stockwell:1997yg}. 

\subsection{Multivariate mode}

Both algorithms estimate a probability value to each data point in the training set, producing  a vector of values in the range [0,1] labelled with the $pa$ presence absence variable in two values ${0,1}$.  While the GLM produces this from an additive model of the odds for each variable, WW2 produces a probability vectors only on single variables at a time.  The two single variables are combined using a fuzzy AND Zadeh operator \cite{Zadeh1965} to produce a new membership vector evaluated again using the ROC or AUC. 

\begin{equation}
AND: x \land y = min(f(x),f(y))
\end{equation}

This eliminates the need to develop and express a higher dimensional model, even though the data is high dimensional, facilitating a data mining approach to very large databases.  

The models for both GLM and 2 can be developed on a test training set and tested on a test set using any protocol.  Here it is  evaluated (using the ROC or AUC). The AUC gives the probability that a model correctly classifies a random draw of a positive and negative example and so is a type of accuracy. In order to avoid over fitting, a K-fold validation is used proceeds by sequentially holding back one fifth of the data each time for evaluation, and developing the model using the remaining four-fifths.   

Analysis was on a Toshiba laptop and were well within the capacity of the machine.  Both were implemented in the R language and all of the code for \texttt{WhyWhere} and for this study is available in R from the development site at http://github.com/davids99us/whywhere.  

\section{Results}

\subsection{Prediction on WorldClim Dataset} 

Predictions by WW2 of the brown-throated three-toed sloth ($Bradypus~variegatus$) were made using 9 variable the WorldClim data set. The variable $bio7$ was selected which is a combination of $bio7$  of $bio7$ = the temperature annual range ($bio5-bio6$) where $bio5$ = maximum temperature of the warmest month and $bio6$ = minimum temperature of the coldest month. Fig~\ref{fig3} shows the $bio7$ and $biome$ variables, the model, and the the predicted distribution over the South American continent with occurrence points. The model lookup table for $bio7$ is shown in Table\ref{tab4}.  This contains ten segments with a width varying from 9 in the segment (108,117] and 115 in the segment (210,max.value]. The table also lists the odds ratio, which varies from 0 where environmental ranges are unoccupied by the species (N=0), to 3.94 in the range from (63,108] of the environmental variable. The probability that a species would be present in that environmental range was 0 to 0.8.  The projection of these probabilities in the geographic space is shown in Figure~\ref{fig1}D.

\begin{figure}[htbp]
\begin{center}
