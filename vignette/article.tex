\documentclass{article}
\usepackage{amsfonts}

\title{WhyWhere: An R  Package for Fuzzy Spatial Modelling}
\author{David Stockwell \hspace{0.4cm} }
\date{}

\usepackage{Sweave}
\begin{document}
\Sconcordance{concordance:article.tex:article.Rnw:%
1 11 1 1 0 34 1 1 2 1 0 1 2 1 0 5 1 3 0 1 2 25 1 1 2 4 0 1 2 104 1}



\maketitle

\begin{abstract}
This introduction to the R package WhyWhere is a rewritten
version of  Stockwell,  which
describes the mathematical basis and philosophy of preidictive and explanatory modelling on big data. The package benifits from recent developments in R like raster which gives access to geograpic operations, dplyr that greatly simplifies the operations on large datasets and also integration with dismo for species distribution modelling.  We show the model is simple and intuitive, efficient to compute, and at equal to the best alternative approaches.  Thus, it makes a powerful tool to utilize large data sets information about species and their environmental relationships and to assess their significance.
\end{abstract}

\section{Introduction} \label{sec:intro}

This package is concerned with the development of interpretable models of self-directed entities such as biological species on large datasets, although the approach could potentially be used in the other fields such as consumer and military intelligence, linguistics and control systems.  The inferential basis is fuzzy set theory, where instead statements that are either true or false, a membership function describes a fuzzy truth value as a function $f: \mathbb{R} \rightarrow to [0,1]$ from a variable $V$ to the real unit interval $[0,1]$.  One must consider membership functions taking values from other spaces such as categories, (also known as factors in R) $N$ or on a space of many variables $V_1 \times V_2 \times ... \times V_n$ where each $V_i$ is an interval in $N$ or $R$. 

Experience has shown that a particularly useful way of combining unitary membership functions to resemble the AND, OR and NOT operators of classical logic are Zadeh operators:

AND: $x \land y = min(f(x),f(y))$, \\
OR: $x \lor y = max(f(x),f(y))$, and \\
NOT: $ \lnot x = (1-f(x))$.  


There have been many approaches to learning fuzzy rules from from given data, and approaches to representation of the discovered rules.  One of the outstanding problems is the trade-off between accuracy and interpretability, or prediction and explanation in the ecological literature.  In particular in this package, ecological theory can motivate our approach, thus satisfying both criteria.  It is the desire to address the problem of explanatory models that motivated t development of WhyWhere -- to describe Why is a species Where? -- without sacrificing predictive accuracy or computational tractability.  

The guide is arranged as follows.  Section 2 describes the main operations, data preparation, developing the membership functions, the operations for building more complex models, and evaluation.  Section 3 compares the performace with the current best methed MaxEnt on a couple of known datasets.

\section{Basic Functions}

\subsection{Data Preparation}

Data are generally obtained as coordinates.  But need in a table with the presence or absence first.  Data preparation from these geocoordinates and environmental variables can proceed  as per the package raster for the environmental variables.  This is usually in the fomr of a RasterStack class.

However, the WhyWhere package adds some functions.  One can prepare a set of rasters with different resolutions but the same crop extent to give a raster pile.  

Here we get the environmental files and read into a raster::RasterStack. 

\begin{Schunk}
\begin{Sinput}
> source("../R/ww1.R")